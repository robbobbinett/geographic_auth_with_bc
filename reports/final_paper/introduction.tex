\section{Motivation}

Geographical authentication
is the task of certify that someone is at a certain place;
that is,
the task of authenticating someone's geographical location.
This problem has been considered for over 20 years~\cite{denning_1996}.
The authors of \cite{denning_1996} consider the problem of finding
the location of an intruder;
other applications include geographically restricted broadcasts~\cite{gdpr},
checkins at Foursquare,
and withdrawing cash at an ATM.

In all these examples,
there is a central authority which confers authentication services.
Besides all the traditional problems with centralized solutions,
like having a single point of failure,
services like Foursquare have to essentially believe the user's word
when attesting they were at a certain place~\cite{glas2015breaking}.
ATMs circumvent this by essentially being a large network of totems,
which is expensive~\cite{totem_patent}.

The above solutions suffer from either having to trust too heavily upon
the user's word, or from implementing an expensive network of totems to
vet user dishonesty. The former makes a system unreliable, while the latter
imposes a startup cost that is not realistic for small players trying to enter
the market.

For reasons of decentralization and low cost, we would like a
means of local geographic authentication where users---each user represented as
a smartphone---mutually and simultaneously encode their geographic location using
handshakes with nearby users.
In this work,
we present a blockchain-like distributed ledger protocol which,
when implemented on top of some
smartphone ad hoc network (SPAN), encodes user geotemporal information strictly in
terms of the topology of that user's local blockchain. We further present a simulation
package which---unlike any preexisting package know to the authors---allows for the
simulation of a blockchain-like ledger system implemented over arbitrary SPAN
topologies.
