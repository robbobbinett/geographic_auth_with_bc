\section{Introduction}

Geographical authentication
is the task of certify that someone is at a certain place;
that is,
the task of authenticating someone's geographical location.
This problem has been considered for over 20 years~\cite{denning_1996}.
The authors of \cite{denning_1996} consider the problem of finding
the location of an intruder;
other applications include geographically restricted broadcasts~\cite{gdpr},
checkins at Foursquare,
and withdrawing cash at an ATM.

In all these examples,
there is a central authority which confers authentication services.
Besides all the traditional problems with centralized solutions,
like having a single point of failure,
services like Foursquare have to essentially believe the user's word
when attesting they were at a certain place.
ATM's circumvent this by essentially being a large network of totems,
which is expensive.

Beyond the scope of simple geographical authentication
are applications in which both geographical data and \textit{post facto}
temporal data are useful, such as
proving someone was or was not present in a certain vicinity within a
certain timeframe.
This is most useful in cases of conflicting information or in cases sensitive to
the falsification of geo-temporal data.


\subsection{Use Cases}

\begin{itemize}
	\item A protester is found dead in a location distant from where the protest happened.
		Police argue the person was not involved in the protest
		and died from unrelated causes.
		Other protesters can use the network to contest police claims
		and show that this specific protester was indeed present in the protest.

	\item A person from a racial minority is being accused of committing a crime.
		This person can use the network to show that the accusation
		is incompatible with the locations that the person has been
		during the time the crime happened.

	\item Some expensive lab equipment disappeared overnight.
		A student not involved with this situation
		can prove her innocence by showing her location during that night
		is far away from the lab.

	\item A professor forgot to take attendance for a class.
		Each student can prove if they were in class
		even after the class has ended.

	\item BBC wants to provide exclusive content for people located in London.
		People can prove that they indeed are that geographical location
		in order to have access to this content.
\end{itemize}

\subsection{Solution Overview}

Essentially,
we use a blockchain over a Smartphone Ad-hoc Network (SPAN).

The ubiquity of smartphones
makes SPANs a cheap and simple underlying infrastructure
that already convey a notion of locality.
The idea is to make users handshake with other users in close geographical proximity,
and record these handshakes in a distributed ledger.

We use blockchains as the distributed ledger.
Traditionally,
blockchain networks strive mantain a single longest path
starting from the root of the chain;
that is,
the goal is to avoid global forks.
In the case of Bitcoin,
for example,
local, temporary forking still happens
but one of those forks is usually quickly abandoned by the network~\cite{decker_2013}.
In our case,
since SPANs are often disconnected,
the global blockchain will eventually fork.
We actually embrace forking as a feature:
each long-term fork represents a connected component of the underlying SPAN.


\subsection{Network Simulation}

In order to observe the behavior of this network under several scenarios,
we wrote a simulator%
\footnote{
	The code for the simulator is publicly available at
	\url{https://github.com/robbobbinett/geographic_auth_with_bc}.
}
for the protocol.

To the best of our knowledge,
this is the first blockchain-like network
that embraces forking as a feature rather than a bug.
Therefore,
there are no packages which are capable of simulating a forking blockchain.
We also are not aware of any packages that implement blockchains over SPANs.

Although we implemented the protocol described in sections
\ref{sec:blockchain} and~\ref{sec:message-passing}
in the simulator,
the package we have written is actually agnostic to the protocol
and even to the underlying node infrastructure.
We hope that these packages can be used by other researchers
to better understand forking blockchains.
