\section{Introduction}

Geographical authentication
is the task of certify that someone is at a certain place;
that is,
the task of authenticating someone's geographical location.
This problem has been considered for over 20 years~\cite{denning_1996}.
The authors of \cite{denning_1996} consider the problem of finding
the location of an intruder;
other applications include geographically restricted broadcasts~\cite{gdpr},
checkins at Foursquare,
and withdrawing cash at an ATM.

In all these examples,
there is a central authority which confers authentication services.
Besides all the traditional problems with centralized solutions,
like having a single point of failure,
services like Foursquare have to essentially believe the user's word
when attesting they were at a certain place~\cite{glas2015breaking}.
ATM's circumvent this by essentially being a large network of totems,
which is expensive~\cite{totem_patent}.

The above solutions suffer from either having to trust too heavily upon
the user's word, or from implementing an expensive network of totems to
vet user dishonesty. The former makes a system unreliable, while the latter
imposes a startup cost that is not realistic for small players trying to enter
the market.

For reasons of decentralization and low cost, we would like a
means of local geographic authentication where users---each user represented as
a smartphone---mutually and simultaneously encode their geographic location using
handshakes with nearby users. In this work, we show that there exists a
blockchain-like distributed ledger protocol which, when implemented on top of some
smartphone ad hoc network (SPAN), encodes user geotemporal information strictly in
terms of the topology of that user's local blockchain. We further present a simulation
package which---unlike any preexisting package know to the authors---allows for the
simulation of a blockchain-like ledger system implemented over arbitrary SPAN
topologies.

\section{The SPANchain Simulator}
\label{sec:SPANchain}

We present a tool which allows for the simulation of the evolution of SPANs
---represented as random geometric graphs---over time, and which, most saliently,
allows for the simulation of a novel blockchain-like distributed ledger overlaid
on this SPAN. While the authors are aware of 1) network simulation tools that
accommodate dynamic network topologies~\cite{chaudhary2012study} and
2) tools for simulating blockchain
systems distributed over a network of nodes, we are unaware of any tool that
allows for the simulation of a blockchain system implemented over arbitrary dynamic
networks. Part of the reason for this is that SPANs, by nature, induce a network topology
that fragments easily and from which nodes enter and exit with great frequency. For
conventional blockchain systems, this inevitably leads to forking in the global
chain---a property which, for cryptocurrency applications, is highly undesirable
~\cite{decker_2013,nakamoto2008bitcoin}. Using our simulation tool, however, we show
that the way global forks form in blockchain networks implemented over SPANs can encode
geotemporal information about how the SPAN evolves over time.

\subsection{Motivation}

Many smartphones today are equipped with wireless functionality specifically
geared towards interacting with other smartphones in near geographic proximity.
Assuming these connections faithfully represent one's geographic
authentication\footnote{This is an important problem which we do not attempt to
address in this paper. We are only concerned with whether a blockchain-like distributed
ledger implemented over a SPAN can reliably encode information about the SPAN's evolution
over time using nothing more than the topology of local copies of the blockchain.},
it follows immediately that, if each smartphone is a node $u$ connected to node $v$ if and only
if $u$ and $v$ are within interaction proximity---the set of neighbors of $u$ serves as
a valid representation of $u$'s geographic location. Further, if two nodes $u,v$ claim to be
in the same geographic location (given the location is sufficiently granular), it is expected
that the set of neighbors they report would significantly overlap. For this reason,
recording ``handshakes'' with peers in geographic proximity should serve as a valid
representation of one's location over time.

\subsection{Three Evolving Topologies: Implications of Blockchain over a SPAN}

\begin{figure*}
	\centering
	\begin{tabular}{|c||c|c|c|}
		\hline
		\ & \textbf{P2P Topology} & \textbf{Local Blockchain (per node)} & \textbf{Global Blockchain} \\
		\hhline{|=||=|=|=|}
		\textbf{Bitcoin P2P} & \ & \ & \ \\
		\hline
		\textbf{SPAN P2P} & \ & \ & \ \\
		\hline
	\end{tabular}
	\caption{The Nakamoto paper assumes that the blockchain protocol is
		implemented over a relatively stable P2P network. If we attempt to
		implement a blockchain protocol over a SPAN, however, many of the
		definitions that naturally arise in the case of Bitcoin fail to arise.}
	\label{tab:three_table}
\end{figure*}

We use a blockchain over a Smartphone Ad-hoc Network (SPAN).

The ubiquity of smartphones
makes SPANs a cheap and simple underlying infrastructure
that already convey a notion of locality.
The idea is to make users handshake with other users in close geographical proximity,
and record these handshakes in a distributed ledger.

We use blockchains as the distributed ledger.
Traditionally,
blockchain networks strive mantain a single longest path
starting from the root of the chain;
that is,
the goal is to avoid global forks.
In the case of Bitcoin,
for example,
local, temporary forking still happens
but one of those forks is usually quickly abandoned by the network~\cite{decker_2013}.
In our case,
since SPANs are often disconnected,
the global blockchain will eventually fork.
We actually embrace forking as a feature:
each long-term fork represents a connected component of the underlying SPAN.


\subsection{Network Simulation}

In order to observe the behavior of this network under several scenarios,
we wrote a simulator%
\footnote{
	The code for the simulator is publicly available at
	\url{https://github.com/robbobbinett/geographic_auth_with_bc}.
}
for the protocol.

To the best of our knowledge,
this is the first blockchain-like network
that embraces forking as a feature rather than a bug.
Therefore,
there are no packages which are capable of simulating a forking blockchain.
We also are not aware of any packages that implement blockchains over SPANs.

Although we implemented the protocol described in sections
\ref{sec:blockchain} and~\ref{sec:message-passing}
in the simulator,
the package we have written is actually agnostic to the protocol
and even to the underlying node infrastructure.
We hope that these packages can be used by other researchers
to better understand forking blockchains.
