\section{Conclusions and Future Work}

% TODO: Write more about simulation results

The protocol, as described in this paper,
is able to provide distributed geographical authentication
An obvious next step would be actually deploying this protocol in a real-life scenario,
to both test its reliability and compute the simulation parameters.

As outlined in Section~\ref{sec:first-order-attacks},
making the protocol resilient to first-order attacks is an open problem.
A possible solution involves adding a computational cost
to enter and stay in the network;
for example,
honest nodes could simply try to solve cryptographic challenges
from nodes which have appeared ``recently'' in the blockchain.

Another issue is that,
in the analysis,
we assume that the computational power of each node
is roughly the same.
This makes the network vulnerable to a smartphone user using,
say,
their personal computer to speed up the computation of hashes
to control the network.
A possible solution is to replace the highly parallizable hash function
with non-parallelizable cryptographic challenges~\cite{Tritilanunt2007}.
