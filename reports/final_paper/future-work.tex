\section{Conclusions and Future Work}

% TODO: Write more about simulation results

We have shown that when a blockchain protocol is implemented over an
\textit{ad hoc} network, it is possible to encode information about the network's evolution within
nodes' copies of the local blockchain. We motivate the utility of using blockchains
to encode this information in the context of local geographic authentication
---fully acknowledging that implementing our protocol to the problem of geographic
authentication requires hardware considerations which we do not address here.
However, given sufficient exploration of the hardware side of the problem,
it is plausible to use the protocol described in this paper
to implement distributed geographical authentication.

More importantly, our work suggests that there exist applications for which
a blockchain that sports ``forking as a feature'' are desirable. Given the
nauseating volume of research currently focused on blockchain technology,
we hope that our findings here---as well as the SPANchain simulator---provide
a platform for blockchain researchers to solve more varied problems using
blockchain variations.

Since using forking as a feature is a novel concept,
there are several questions regarding its behavior over a SPAN
which have no answer yet.
For example,
we may enquire, over a given SPAN, the average radius from a source node $u$
for which non-trivial semi-global blockchains cease to exist. We may also wish
to formalize means of counting discrete forks, or otherwise making an indiscrete
forking index, for applications wherein ``forking as a feature'' is desireable.
Tools for analyzing and visualizing these forks, of course, would also be 
required in such a wave of next steps.

As outlined in Section~\ref{sec:first-order-attacks},
making the protocol resilient to first-order attacks is an open problem.
A possible solution involves adding a computational cost
to enter and stay in the network;
for example,
honest nodes could simply try to solve cryptographic challenges
from nodes which have appeared ``recently'' in the blockchain.
