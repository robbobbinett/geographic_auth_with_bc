\documentclass{article}

\usepackage[T1]{fontenc}
\usepackage[utf8]{inputenc}

\usepackage{amsmath}
\usepackage{amssymb}
\usepackage{amsthm}

\newtheorem{proposition}{Proposition}

\begin{document}

\title{Frequent-Collision Blockchains for Local Geographic Authentication}
\author{Ryan Robinett and Tiago Royer}
\date{11 Dec 2019}
\maketitle

\begin{abstract}
	Blockchains are an implementation of a public distributed ledger,
	which allows for a certain information to be agreed upon by several parties
	without the need of a central authority.
	This work adapts this distributed ledger
	to implement local geographical authentication.
	By forcing the global blockchain to fork,
	each maximal chain is shared by only a few nodes in close geographical proximity,
	so each geographical location is represented by a maximal chain.
	The distributed nature of the blockchain
	prevents issues stemming from centralized authorities.
\end{abstract}

\section{Introduction}

\section{Blockchain for Geographical Authentication}

\section{Message-passing algorithm}

\section{Simulation}
\subsection{Poisson law for block generation}
\subsection{SPAN Connectivity Model}

The protocol runs over an ad-hoc network,
where each node is a smartphone
and the nodes connect wirelessly to each other.
In the simulation,
we modelled this underlying infrastructure using a dynamic random graph.

The most commonly studied model of random graphs is the Erdős--Rényi model.
Given parameters $n$ and $p$,
the model generates a graph with $n$ vertices
and each edge is added to the graph independently with probability $p$~\cite{bollobas_2001}.
This model, however,
has no information about node positions,
which makes it a poor representative of ad-hoc networks~\cite{Hekmat2006}.

Instead,
we will use random geometric graphs as our connectivity model.
Given parameters $n$ and $r$,
the model randomly chooses $n$ points in the unit square $[0, 1]^2$,
uniformly and independently,
and adds an edge between each points which are at a distance $r$ from each other.
Random geometric graphs
have been proposed as accurate models for ad-hoc networks~\cite{Kenniche2010,Hekmat2006}.

Nodes which are close to the borders of the unit square
are affected by the so-called ``border effect'':
if the distance of a node is smaller than $r$
from any of the borders,
then part of the ``area of coverage'' for this node
(that is,
the area which may contain adjacent vertices)
lies outside of the unit square.
This effectively truncates the degree of that node,
because no nodes are generated outside the unit square.
For smaller values of $r$ this effect is less pronounced,
because less nodes are affected.

In order to avoid the border effect,
we will use toroidal distances instead~\cite{Kenniche2010,Hekmat2006}.
The nodes will be placed on the unit torus $[0, 1)^2$
and distances will be measured according to a toroidal metric.
So,
for example,
for $r = 0.1$,
the nodes at $(0.01, 0.01)$ and $(0.99, 0.99)$ are adjacent in this metric.

We will denote
the toroidal random geometric graph model with parameters $n$ and $r$
by $T(n, r)$.

We want to model the underlying infrastructure to be dynamic,
to represent the fact that nodes move around in the world.
That is,
we will have a sequence $G_0, G_1, \dots$ of graphs over the same set of nodes,
which represent the evolution of the network.

Using the toroidal has the benefit that
if the nodes move around randomly, but independently,
then each resulting graph is still a geometric graph.
More precisely,

\begin{proposition}
	Let $G_0 \in T(n, r)$ be a random graph.
	For each $i \in \mathbb N$,
	define $G_{i+1}$ by translating each vertex $v \in V(G_i)$
	by a random translation vector $u_v$,
	chosen independently according to some distribution,
	and connecting vertices which are within a distance of $r$ of each other.
	Then, for each $i$,
	the graph $G_i$ is a random geometric graph distributed according to $T(n, r)$.
\end{proposition}

Note that each $G_i$ will be distributed according to $T(n, r)$,
regardless of the distribution of the translation vectors $u_v$
(as long as each translation vector is chosen independently,
and irrespective of the vector $v$).

% TODO: Proof?

However,
there are little literature using several different translation parameters.
We will use a simplified version of the dynamic model of~\cite{Diaz2008}.
Fixed a parameter $s$,
$G_{i+1}$ is generated from $G_i$
by translating each vertex by a vector with norm $s$,
chosen uniformly among all $s$-normed vectors in $\mathbb R^2$.

% TODO: Picture?


\section{Resilience against first-order attacks}

Thoughoug the entire paper,
we made the assumption that all nodes have about the same computational power.
In this section,
we consider the situation in which a single node,
without additional computational power,
is trying to disrupt the network.
That is,
we analyze the protocol's resiliency against first-order attacks.

A misbehaving node may only interfere with either the blockchain infrastructure
or with the message-passing algorithm.

Since it is not possible to erase any blocks from the blockchain,
the only way a misbehaving node could affect the ledger
is by adding blocks.
However,
as we are supposing the malicious node
has about the same computational power as the other nodes in the network,
the villain would only be able to add blocks at the same rate as the other nodes,
so these extraneous blocks would be a minority in the chain.
Distributed consensus cannot be affected by a few blocks in the chain.

For the message-passing algorithm,
the malicious node cannot forge or alter messages from other users
due to cryptographic signatures.
Dropping messages from other users is not effective;
this is tantamount to making the network disconnected,
which is a situation the protocol is designed to tolerate.

However,
the node is able to generate several extraneous cryptographic challenges;
that is,
a single node could create several virtual identities
(by generating several pairs of public and private keys)
and pose cryptographic challenges to other nodes posing as each of these identities.
Other nodes in the network would not be able to differentiate between
these virtual identities and other honest nodes,
and thus waste cycles trying to solve cryptographic challenges
posed by nodes which do not exist.
Although this attack does not erase previous blocks,
it slows down the network,
preventing it from doing meaningful progress.



\section{Conclusion and open problems}

\bibliographystyle{plain}
\bibliography{bib}

\end{document}
