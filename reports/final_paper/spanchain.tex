\section{The SPANchain Simulator}
\label{sec:SPANchain}

We present a tool which allows for the simulation of the evolution of SPANs%
---represented as random geometric graphs---over time, and which, most saliently,
allows for the simulation of a novel blockchain-like distributed ledger overlaid
on this SPAN. While the authors are aware of 1) network simulation tools that
accommodate dynamic network topologies~\cite{chaudhary2012study} and
2) tools for simulating blockchain
systems distributed over a network of nodes, we are unaware of any tool that
allows for the simulation of a blockchain system implemented over arbitrary dynamic
networks. Part of the reason for this is that SPANs, by nature, induce a network topology
that fragments easily and from which nodes enter and exit with great frequency. For
conventional blockchain systems, this inevitably leads to forking in the global
chain---a property which, for cryptocurrency applications, is highly undesirable
~\cite{decker_2013,nakamoto2008bitcoin}. Using our simulation tool, however, we show
that the way global forks form in blockchain networks implemented over SPANs can encode
geotemporal information about how the SPAN evolves over time.

In order to observe the behavior of this network under several scenarios,
we wrote a simulator package SPANchain%
\footnote{
	The code for the simulator is publicly available at
	\url{https://github.com/robbobbinett/geographic_auth_with_bc}.
}
for the protocol.

To the best of our knowledge,
this is the first blockchain simulation framework which simulates
blockchains implemented over SPANs, as well as the first
to embrace forking as a feature rather than a bug.
We also are not aware of any packages that implement blockchains over SPANs.

Although we implemented the protocol
described in section~\ref{sec:blockchain}
in the simulator,
the package is written to be agnostic to the protocol for block formation,
as well as processes by which the underlying topology evolves.%
\footnote{
	We have, however,
	implemented a class of SPAN-nodes
	to specifically handle the evolution of the SPAN
	with respect to a random geometric graph model.
}
We hope that these packages can be used by other researchers
to better understand how blockchains fork, and what information this forking
encodes with respect to the history of the underlying topology.
