\section{Adapting the Bitcoin Protocol}
\label{sec:blockchain}

This section formalizes how the SPANchain module models the Bitcoin protocol
for block creation and validation. This adaptation loses the notion of
a transaction---the primitive stored by the blocks in Bitcoin---and convolves the notion
of a block and handshake into essentially the same primitive. This adaptation
pivots from the Bitcoin protocol in such a way as to make this convolution
well-defined while preserving some of the robustness of the blockchain against
adversarial behavior shown by Nakamoto~\cite{nakamoto2008bitcoin}.

\subsection{Exchanging Transactions for Handshakes}

The blockchain is essentially a tree of blocks,
containing solutions to cryptographic challenges.
A \emph{cryptographic challenge} is a pair
\begin{equation*}
	\mathcal P = (t_C, k_C)
\end{equation*}
where $t$ is a timestamp and $k$ is the public key of the problem creator.
Each \emph{block} is, in turn, a tuple of the form
\begin{equation*}
	B = (t_C, k_C, t_S, k_S, n, b),
\end{equation*}
where $(t_C, k_C)$ form a cryptographic challenge,
$n$ is the nonce which solves the challenge (described below)
$k_S$ is the public key of the problem solver,
$t_S$ is the timestamp of when $n$ was found,
and $b$ is a pointer to a parent block.
Additionally,
the empty block $(\varnothing, \varnothing, \varnothing, \varnothing, \varnothing)$
is a valid block.
This is the global root of the blockchain tree.

We assume the existence of a cryptographically secure hash function $H$.
A nonce $n$ is a \emph{solution} to the problem contained in a block $B$ as above
if $H(B) < \tau$,
where $\tau$ is a predefined threshold value known \emph{a priori} by the network.

\subsection{Rate of Block Formation as a Function of $\tau$}

Nakamoto's original paper discusses regulating the rate of block formation in
the global chain by
periodically updating the threshold $\tau$ for which all blocks $B$
must satisfy $H(B)<\tau$~\cite{nakamoto2008bitcoin}.
Decker \textit{et al.} subsequently show that the rate at which block collisions
occur in the Bitcoin network can be modeled as a function of the rate of block
creation. Putting these together, we surmise that our rates of block formation
and global forking can be set arbitrarily high or low relative to the rates at
which changes occur in the topology of the underlying SPAN. This allows us to
simulate our peer-to-peer (P2P) message passing over the SPAN in such
a way that, given network churn is kept low, we can run SPAN updates and
block updates/propagation as discrete events that do not overlap or otherwise
interfere.

% TODO: add example
