\section{Blockchain for Geographical Authentication}
\label{sec:blockchain}

Once a block is generated and published,
it is never removed from the blockchain;
so, in a sense,
blockchains are static objects.
This section describes the static features of the blockchain
that is used as the distributed ledger in our protocol.

The blockchain is essentially a tree of blocks,
containing solutions to cryptographic challenges.
A \emph{cryptographic challenge} is a pair
\begin{equation*}
	\mathcal P = (t_C, k_C)
\end{equation*}
where $t$ is a timestamp and $k$ is the public key of the problem creator.
Each \emph{block} is, in turn, a tuple of the form
\begin{equation*}
	B = (t_C, k_C, t_S, k_S, n, b),
\end{equation*}
where $(t_C, k_C)$ form a cryptographic challenge,
$n$ is the nonce which solves the challenge (described below)
$k_S$ is the public key of the problem solver,
$t_S$ is the timestamp of when $n$ was found,
and $b$ is a pointer to another block.
Additionally,
the empty block $(\varnothing, \varnothing, \varnothing, \varnothing, \varnothing)$
is a valid block.
This is the global root of the blockchain tree.

We assume the existence of a cryptographically secure hash function $H$.
Denote by $t_c k_C k_S b$ the concatenation of $t_c$, $k_C$, $k_S$ and $b$.
A nonce $n$ is a \emph{solution} to the problem contained in a block $B$ as above
if $H(t_C k_C k_S b) < \tau$,
where $\tau$ is a predefined threshold value known \emph{a priori} by the network.

% TODO: add example
