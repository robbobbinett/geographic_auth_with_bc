\section{Resilience against first-order attacks}
\label{sec:first-order-attacks}

In order to analyze adversarial behavior,
we make the assumption that all nodes have about the same computational power.
We consider the situation in which a single node,
without additional computational power,
is trying to disrupt the network.
That is,
we analyze the protocol's resilience against first-order attacks.

A misbehaving node may only interfere with either the blockchain infrastructure
or with the message-passing algorithm.

Since it is not possible to erase any blocks from the blockchain,
the only way a misbehaving node could affect the ledger
is by adding blocks.
However,
as we are supposing the malicious node
has about the same computational power as the other nodes in the network,
the villain would only be able to add blocks at the same rate as the other nodes,
so these extraneous blocks would be a minority in the chain.
Distributed consensus cannot be affected by a few blocks in the chain.

For the message-passing algorithm,
the malicious node cannot forge or alter messages from other users
due to cryptographic signatures.
Dropping messages from other users is not effective;
this is tantamount to making the network disconnected,
which is a situation the protocol is designed to tolerate.

However,
the node is able to generate several extraneous cryptographic challenges;
that is,
a single node could create several virtual identities
(by generating several pairs of public and private keys)
and pose cryptographic challenges to other nodes posing as each of these identities.
Other nodes in the network would not be able to differentiate between
these virtual identities and other honest nodes,
and thus waste cycles trying to solve cryptographic challenges
posed by nodes which do not exist.
Although this attack does not erase previous blocks,
it slows down the network,
preventing it from doing meaningful progress.
