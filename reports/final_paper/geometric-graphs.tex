\subsection{SPAN Connectivity Model}

The protocol runs over an ad-hoc network,
where each node is a smartphone
and the nodes connect wirelessly to each other.
In the simulation,
we modelled this underlying infrastructure using a dynamic random graph.

The most commonly studied model of random graphs is the Erdős--Rényi model.
Given parameters $n$ and $p$,
the model generates a graph with $n$ vertices
and each edge is added to the graph independently with probability $p$~\cite{bollobas_2001}.
This model, however,
has no information about node positions,
which makes it a poor representative of ad-hoc networks~\cite{Hekmat2006}.

Instead,
we will use random geometric graphs as our connectivity model.
Given parameters $n$ and $r$,
the model randomly chooses $n$ points in the unit square $[0, 1]^2$,
uniformly and independently,
and adds an edge between each points which are at a distance $r$ from each other.
Random geometric graphs
have been proposed as accurate models for ad-hoc networks~\cite{Kenniche2010,Hekmat2006}.

Nodes which are close to the borders of the unit square
are affected by the so-called ``border effect'':
if the distance of a node is smaller than $r$
from any of the borders,
then part of the ``area of coverage'' for this node
(that is,
the area which may contain adjacent vertices)
lies outside of the unit square.
This effectively truncates the degree of that node,
because no nodes are generated outside the unit square.
For smaller values of $r$ this effect is less pronounced,
because less nodes are affected.

In order to avoid the border effect,
we will use toroidal distances instead~\cite{Kenniche2010,Hekmat2006}.
The nodes will be placed on the unit torus $[0, 1)^2$
and distances will be measured according to a toroidal metric.
So,
for example,
for $r = 0.1$,
the nodes at $(0.01, 0.01)$ and $(0.99, 0.99)$ are adjacent in this metric.

We will denote
the toroidal random geometric graph model with parameters $n$ and $r$
by $T(n, r)$.

We want to model the underlying infrastructure to be dynamic,
to represent the fact that nodes move around in the world.
That is,
we will have a sequence $G_0, G_1, \dots$ of graphs over the same set of nodes,
which represent the evolution of the network.

Using the toroidal has the benefit that
if the nodes move around randomly, but independently,
then each resulting graph is still a geometric graph.
More precisely,

\begin{proposition}
	Let $G_0 \in T(n, r)$ be a random graph.
	For each $i \in \mathbb N$,
	define $G_{i+1}$ by translating each vertex $v \in V(G_i)$
	by a random translation vector $u_v$,
	chosen independently according to some distribution,
	and connecting vertices which are within a distance of $r$ of each other.
	Then, for each $i$,
	the graph $G_i$ is a random geometric graph distributed according to $T(n, r)$.
\end{proposition}

Note that each $G_i$ will be distributed according to $T(n, r)$,
regardless of the distribution of the translation vectors $u_v$
(as long as each translation vector is chosen independently,
and irrespective of the vector $v$).

% TODO: Proof?

However,
there are little literature using several different translation parameters.
We will use a simplified version of the dynamic model of~\cite{Diaz2008}.
Fixed a parameter $s$,
$G_{i+1}$ is generated from $G_i$
by translating each vertex by a vector with norm $s$,
chosen uniformly among all $s$-normed vectors in $\mathbb R^2$.

% TODO: Picture?
