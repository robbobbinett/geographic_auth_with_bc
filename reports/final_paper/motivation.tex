\section{Blockchains over SPANs}
\label{sec:motivation}

Many smartphones today are equipped with wireless functionality specifically
geared towards interacting with other smartphones in near geographic proximity.
Assuming these connections faithfully represent geographic proximity%
\footnote{
	Breaking this assumption is an important problem which we do not attempt to
	address in this paper. We are only concerned with whether a blockchain-like distributed
	ledger implemented over a SPAN can reliably encode information about the SPAN's evolution
	over time using nothing more than the topology of local copies of the blockchain.
},
it follows immediately that, if each smartphone is a node $u$ connected to node $v$ if and only
if $u$ and $v$ are within interaction distance---the set of neighbors of $u$ serves as
a valid representation of $u$'s geographic location. Further, if two nodes $u,v$ claim to be
in the same geographic location (given the location is sufficiently granular), it is expected
that the set of neighbors they report would significantly overlap. For this reason,
recording ``handshakes'' with peers in geographic proximity should serve as a valid
representation of one's location over time.

\subsection{Three Evolving Topologies: Implications of Blockchain over a SPAN}

The Bitcoin protocol is heavily dependent on a few assumptions concerning
the topology of the underlying network. While it relies on the fact that no
node can have global knowledge of what nodes are in the network due to the
ease with which nodes may enter and exit the network, it also assumes that the
network is stable enough to offer such features as never partitioning and never
being susceptible to man-in-the-middle attacks~\cite{nakamoto2008bitcoin}. This
relative stability is what allows all nodes in the network to asymptotically
agree on which blocks constitute the global blockchain. Further, this global
blockchain naturally arises as the set-theoretic intersection of all local 
chains in the network.

Implementing a blockchain system over a SPAN, however, breaks all of the
aforementioned assumptions. SPANs are highly volatile, and they can easily
partition into disconnected components. This makes the idea of nodes
globally converging to agreement on what is a global blockchain ridiculous, as
the intersection of all local blockchains would cease to grow as soon as
nodes disagree on which is the longest chain. It is guaranteed, however,
that for all nodes $u$ in the SPAN $G$, there exists some neighborhood
$U\subset G$ containing $u$ such that, if one takes the intersection
\begin{equation*}
	\mathbf{B}_U = \bigcap_{v \in U}\mathcal{B}_v
\end{equation*}
of all local blockchains $\mathcal{B}_v$ for all $v \in U$,
it holds that $\mathbf{B}_U$ grows nontrivially for all timesteps%
\footnote{
	To see that this is correct,
	it is sufficient to note that this trivially holds for $U=\{u\}$.
}.

The degree to which all subsets $U\subset G$ satisfy this property is the
degree to which a global blockchain is well defined, and the degree to which
most subsets $U$ fail to do so can serve as a measure of the degree to which
a global blockchain is poorly defined. For this reason, the authors present
the idea of a \textit{semi-global blockchain}: the global blockchain with regards
to a connected subnetwork $U$ of the network $G$, as arises
as the set-theoretic intersection of all local chains $\mathcal{B}_v$ held by
nodes $v \in U$.

\begin{figure*}[t]
	\centering
	\begin{tabular}{|p{3cm}||p{6.7cm}|p{6.7cm}|}
		\hline
		& \textbf{Bitcoin P2P} & \textbf{SPAN P2P} \\
		\hhline{|-||-|-|}

		\textbf{P2P Topology} &
			The Bitcoin protocol relies on the assumption that
			no node in the underlying network can have
			complete global knowledge of the state of the
			network, especially given nodes can spontaneously enter or
			exit the network. However, it is also assumed that
			this network never becomes disconnected and never
			becomes vulnerable to man-in-the-middle attacks.
			In this way, it is safe to assume that all local
			nodes can eventually agree on some global state
			information.
			&
			SPANs are highly volatile. The frequent making and
			breaking of connections causes the SPAN
			to fragment into disconnected components. It is hopeless,
			therefore, to assume that all the nodes will ever
			agree on global state information that is not
			granted \textit{a priori}.
			\\
		\hline

		\raggedright
		\textbf{Local Block-\\chain (per node)} & % Kludge, I know
			\multicolumn{2}{|l|}{
				Very similar, up to slight differences in protocol
			}
			\\
		\hline

		\textbf{Global Blockchain} &
			The global blockchain is implicitly induced as the
			set-theoretic intersection of all local blockchains.
			It is guaranteed in probability that this intersection
			chain is always growing; failure to grow would imply
			that this global chain has forked.
			&
			Though the set-theoretic intersection of all local
			blockchains is well-defined, this intersection is
			likely to be noninteresting and stop growing after some
			timestep due to forking.
			\\
		\hline

		\raggedright
		\textbf{Semi-Global Blockchain (per neighborhood)} &
			For any connected subnetwork $G^\prime$ of the network, the
			global chain of the restriction to $G^\prime$ always
			(asymptotically) agrees with the well-defined global
			chain.
			&
			Let $G$ be the network. For each node $u\in G$, there exists
			neighborhoods $U,u\in U\subset G$ such that the global
			chain is well defined with respect to the restriction
			to $U$. We can define \textit{semi-global blockchains}
			as the intersection of local chains with respect to
			such restrictions.
			\\
		\hline
	\end{tabular}
	\caption{The Nakamoto paper assumes that the blockchain protocol is
		implemented over a relatively stable P2P network. If we attempt to
		implement a blockchain protocol over a SPAN, however, many of the
		notions that naturally arise in the case of Bitcoin no longer appear.
		The SPANchain simulator was written to allow for easy simulation of 
		blockchains implemented over SPANs, together with graphic tools which
		the authors hope helps the research community come up with ways
		of studying semi-global blockchains in the case of a forking global chain.}
	\label{tab:three_table}
\end{figure*}

Traditionally,
blockchain networks strive mantain a single longest path
starting from the root of the chain;
that is,
the goal is to avoid global forks.
In the case of Bitcoin,
for example,
local, temporary forking still happens
but one of those forks is usually quickly abandoned by the network~\cite{decker_2013}.
In our case,
since SPANs are often disconnected,
the global blockchain will eventually fork.
We actually embrace forking as a feature:
each long-term fork represents a connected component of the underlying SPAN.
