\section{Resilience against first-order attacks}

Thoughoug the entire paper,
we made the assumption that all nodes have about the same computational power.
In this section,
we argue that a single node,
without additional computational power,
cannot disrupt the network.
In other words,
this protocol is resilient against first-order attacks.

A misbehaving node may only interfere with either the blockchain infrastructure
or with the message-passing algorithm.

Since it is not possible to erase any blocks from the blockchain,
the only way a misbehaving node could affect the ledger
is by adding blocks.
However,
as we are supposing the malicious node
has about the same computational power as the other nodes in the network,
the villain would only be able to add blocks at the same rate as the other nodes,
so these extraneous blocks would be a minority in the chain.
Distributed consensus cannot be affected by a few blocks in the chain.

For the message-passing algorithm,
the malicious node cannot forge or alter messages due to cryptographic signatures.
Thus,
the worst the node can do is simply refuse to propagate messages from other users.
This is tantamount to making the network disconnected,
which is a situation the protocol is designed to tolerate.
% TODO: generate extraneous cryptographic challenges?
