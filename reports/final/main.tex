\documentclass{article}

\usepackage[T1]{fontenc}
\usepackage[utf8]{inputenc}

\usepackage{amsmath}
\usepackage{amssymb}
\usepackage{amsthm}

\newtheorem{proposition}{Proposition}

\begin{document}

\title{Frequent-Collision Blockchains for Local Geographic Authentication}
\author{Ryan Robinett and Tiago Royer}
\date{11 Dec 2019}
\maketitle

\begin{abstract}
	Blockchains are an implementation of a public distributed ledger,
	which allows for a certain information to be agreed upon by several parties
	without the need of a central authority.
	This work adapts this distributed ledger
	to implement local geographical authentication.
	By forcing the global blockchain to fork,
	each maximal chain is shared by only a few nodes in close geographical proximity,
	so each geographical location is represented by a maximal chain.
	The distributed nature of the blockchain
	prevents issues stemming from centralized authorities.
\end{abstract}

\section{Introduction}

The problem of authenticating users based on geographical location
has been considered for over 20 years~\cite{denning_1996}.
The authors of \cite{denning_1996} consider the problem of finding
the location of an intruder;
their approach easily generalizes to such applications as
geographically restricted broadcasts~\cite{gdpr}.

Beyond the scope of simple geographical authentication, however,
are applications in which both geographical data and \textit{post facto}
temporal data are useful, such as
proving someone was or was not present in a certain vicinity within a
certain timeframe.
This is most useful in cases of conflicting information or in cases sensitive to
the falsification of geo-temporal data.

\subsection{Use Cases}

\begin{itemize}
	\item A protester is found dead in a location distant from where the protest happened.
		Police argue the person was not involved in the protest
		and died from unrelated causes.
		Other protesters can use the network to contest police claims
		and show that this specific protester was indeed present in the protest.

	\item A person from a racial minority is being accused of committing a crime.
		This person can use the network to show that the accusation
		is incompatible with the locations that the person has been
		during the time the crime happened.

	\item Some expensive lab equipment disappeared overnight.
		A student not involved with this situation
		can prove her innocence by showing her location during that night
		is far away from the lab.

	\item A professor forgot to take attendance for a class.
		Each student can prove if they were in class
		even after the class has ended.

	\item BBC wants to provide exclusive content for people located in London.
		People can prove that they indeed are that geographical location
		in order to have access to this content.
\end{itemize}

\subsection{Solution Overview}

In order to meet the requirement of not having a centralized authority,
we propose the use of a distributed ledger to record data in the network.
In particular, we will use blockchains.

Blockchain networks behave as probabilistic state machines, as no nodes in the
network (i.e. no node within the graph corresponding to an individual state) has
general knowledge of the rest of the network/state and must act probabilistically
based on certain assumptions about the network~\cite{saito_2016}. Despite this
state machine construction, blockchain networks strive to mitigate the
possibility of a network fork as much as possible. In the case of Bitcoin, this
avoidance has been successful since the blockchain's inception~\cite{decker_2013}.
Bitcoin avoids forks and maintains a single long-term history of transactions
(i.e. the longest path from the root node to the leaf on the longest branch), as
well as controls the rate of growth of the blockchain, by carefully moderating
the rate of block creation.

While the Bitcoin network moderates block creation so as to create a single,
decentralized ledger for all Bitcoin transactions for all of time, we seek to use
the blockchain dynamics described in \cite{decker_2013} to create a
blockchain-like decentralized ledger which records close-proximity peer-to-peer
handshakes as transactions and which creates blocks of these transactions at such
a rate as to force the network to fork into local networks. The intent is to
create a decentralized ledger of peer-to-peer interactions that
\begin{enumerate}
	\item creates blocks and transactions at such an exacerbatory rate that
		it is unfeasible for any node to cache a chain of blocks for
		longer than a period of one or two days; and
	\item creates blocks and transactions at such an exacerbatory rate that
		users are incentivized to only cache blocks necessary to verify
		handshakes they have made over the cache-life of a block.
\end{enumerate}

This will allow people to create a robust ledger, whose
credibility is granted by mass consensus and mutual distrust, of their temporo-geographic
location over short periods of time, in such a way that makes long-term storage of these
geographic data untenable by third parties while making sure short-term
information about location is agreed upon by several nodes in the network.


\section{Adapting the Bitcoin Protocol}
\label{sec:blockchain}

\subsection{Exchanging Transactions for Handshakes}

Once a block is generated and published,
it is never removed from the local blockchain of a node recognizing
this block. The Bitcoin network ensures this feature with high
probability by forcing a child block to reference a pre-existing parent
block using a nonce $n$ which, together with a hashed parent ID $P$, satisfies
the inequality $H(n,P)<\tau$ for globally define hash function $H$ and
threshold value $\tau$~\cite{nakamoto2008bitcoin}.
This section formalizes how blocks are created and recognized by nodes in a SPAN
in the SPANchain module.

The blockchain is essentially a tree of blocks,
containing solutions to cryptographic challenges.
A \emph{cryptographic challenge} is a pair
\begin{equation*}
	\mathcal P = (t_C, k_C)
\end{equation*}
where $t$ is a timestamp and $k$ is the public key of the problem creator.
Each \emph{block} is, in turn, a tuple of the form
\begin{equation*}
	B = (t_C, k_C, t_S, k_S, n, b),
\end{equation*}
where $(t_C, k_C)$ form a cryptographic challenge,
$n$ is the nonce which solves the challenge (described below)
$k_S$ is the public key of the problem solver,
$t_S$ is the timestamp of when $n$ was found,
and $b$ is a pointer to a parent block.
Additionally,
the empty block $(\varnothing, \varnothing, \varnothing, \varnothing, \varnothing)$
is a valid block.
This is the global root of the blockchain tree.

We assume the existence of a cryptographically secure hash function $H$.
Denote by $t_c k_C k_S b$ the concatenation of $t_c$, $k_C$, $k_S$ and $b$.
A nonce $n$ is a \emph{solution} to the problem contained in a block $B$ as above
if $H(B) < \tau$,
where $\tau$ is a predefined threshold value known \emph{a priori} by the network.

\subsection{Rate of Block Formation as a Function of $\tau$}

Nakamoto's original paper discusses regulating the rate of block formation in
the global chain by
periodically updating the threshold $\tau$ for which all blocks
$B=(t_C, k_C, t_S, k_S, n, b)$ must satisfy $H(B)<\tau$~\cite{nakamoto2008bitcoin}.
Decker \textit{et al.} subsequently show that the rate at which block collisions
occur in the Bitcoin network can be modeled as a function of the rate of block
creation. Putting these together, we surmise that our rates of block formation
and global forking can be set arbitrarily high or low relative to the rates at
which changes occur in the topology of the underlying SPAN. This allows us to
simulate our peer-to-peer (P2P) message passing over the SPAN in such
a way that, given network churn is kept low, we can run SPAN updates and
block updates/propagation as discrete events that do not overlap or otherwise
interfere.

% TODO: add example


\section{Message-passing algorithm}

\section{Simulation}
\subsection{Poisson law for block generation}
\subsection{SPAN Connectivity Model}

The protocol runs over an ad-hoc network,
where each node is a smartphone
and the nodes connect wirelessly to each other.
In the simulation,
we modelled this underlying infrastructure using a dynamic random graph.

The most commonly studied model of random graphs is the Erdős--Rényi model.
Given parameters $n$ and $p$,
the model generates a graph with $n$ vertices
and each edge is added to the graph independently with probability $p$~\cite{bollobas_2001}.
This model, however,
has no information about node positions,
which makes it a poor representative of ad-hoc networks~\cite{Hekmat2006}.

Instead,
we will use random geometric graphs as our connectivity model.
Given parameters $n$ and $r$,
the model randomly chooses $n$ points in the unit square $[0, 1]^2$,
uniformly and independently,
and adds an edge between each points which are at a distance $r$ from each other.
Random geometric graphs
have been proposed as accurate models for ad-hoc networks~\cite{Kenniche2010,Hekmat2006}.

Nodes which are close to the borders of the unit square
are affected by the so-called ``border effect'':
if the distance of a node is smaller than $r$
from any of the borders,
then part of the ``area of coverage'' for this node
(that is,
the area which may contain adjacent vertices)
lies outside of the unit square.
This effectively truncates the degree of that node,
because no nodes are generated outside the unit square.
For smaller values of $r$ this effect is less pronounced,
because less nodes are affected.

In order to avoid the border effect,
we will use toroidal distances instead~\cite{Kenniche2010,Hekmat2006}.
The nodes will be placed on the unit torus $[0, 1)^2$
and distances will be measured according to a toroidal metric.
So,
for example,
for $r = 0.1$,
the nodes at $(0.01, 0.01)$ and $(0.99, 0.99)$ are adjacent in this metric.

We will denote
the toroidal random geometric graph model with parameters $n$ and $r$
by $T(n, r)$.

We want to model the underlying infrastructure to be dynamic,
to represent the fact that nodes move around in the world.
That is,
we will have a sequence $G_0, G_1, \dots$ of graphs over the same set of nodes,
which represent the evolution of the network.

Using the toroidal has the benefit that
if the nodes move around randomly, but independently,
then each resulting graph is still a geometric graph.
More precisely,

\begin{proposition}
	Let $G_0 \in T(n, r)$ be a random graph.
	For each $i \in \mathbb N$,
	define $G_{i+1}$ by translating each vertex $v \in V(G_i)$
	by a random translation vector $u_v$,
	chosen independently according to some distribution,
	and connecting vertices which are within a distance of $r$ of each other.
	Then, for each $i$,
	the graph $G_i$ is a random geometric graph distributed according to $T(n, r)$.
\end{proposition}

Note that each $G_i$ will be distributed according to $T(n, r)$,
regardless of the distribution of the translation vectors $u_v$
(as long as each translation vector is chosen independently,
and irrespective of the vector $v$).

% TODO: Proof?

However,
there is nothing cohesive in the literature about using several different
translation parameters at once.
We will use a simplified version of the dynamic model of~\cite{Diaz2008}.
Fixed a parameter $s$,
$G_{i+1}$ is generated from $G_i$
by translating each vertex by a vector with norm $s$,
chosen uniformly among all $s$-normed vectors in $\mathbb R^2$.


\section{Resilience against first-order attacks}
\label{sec:first-order-attacks}

In order to analyze adversarial behavior,
we make the assumption that all nodes have about the same computational power.
We consider the situation in which a single node,
without additional computational power,
is trying to disrupt the network.
That is,
we analyze the protocol's resilience against first-order attacks.

A misbehaving node may only interfere with either the blockchain infrastructure
or with the message-passing algorithm.

Since it is not possible to erase any blocks from the blockchain,
the only way a misbehaving node could affect the ledger
is by adding blocks.
However,
as we are supposing the malicious node
has about the same computational power as the other nodes in the network,
the villain would only be able to add blocks at the same rate as the other nodes,
so these extraneous blocks would be a minority in the chain.
Distributed consensus cannot be affected by a few blocks in the chain.

For the message-passing algorithm,
the malicious node cannot forge or alter messages from other users
due to cryptographic signatures.
Dropping messages from other users is not effective;
this is tantamount to making the network disconnected,
which is a situation the protocol is designed to tolerate.

However,
the node is able to generate several extraneous cryptographic challenges;
that is,
a single node could create several virtual identities
(by generating several pairs of public and private keys)
and pose cryptographic challenges to other nodes posing as each of these identities.
Other nodes in the network would not be able to differentiate between
these virtual identities and other honest nodes,
and thus waste cycles trying to solve cryptographic challenges
posed by nodes which do not exist.
Although this attack does not erase previous blocks,
it slows down the network,
preventing it from doing meaningful progress.



\section{Conclusion and open problems}

\bibliographystyle{plain}
\bibliography{bib}

\end{document}
