\documentclass{article}
\usepackage[margin=1in]{geometry}

\usepackage[utf8]{inputenc}
\usepackage[T1]{fontenc}

\usepackage{amsmath}
\usepackage{amssymb}
\usepackage{amsthm}

\usepackage[moderate]{savetrees}

\title{Frequent-Collision Blockchains for Local Geographic Authentication}
\author{Ryan Robinett and Tiago Royer}
\date{9 Nov 2019}
\begin{document}

\maketitle

\section*{Preliminary Definitions}
\begin{description}
	\item[\textbf{Public Key:}] If $C$ is a node in the network with unique public key
		$\mathcal{C}$, we will refer to $C$ using both ``$C$'' and ``$\mathcal{C}$''
		interchangeably.
	\item[\textbf{Problem Proposal:}] A problem proposal $(t_a,\mathcal{C},\mathcal{P})$
		is a tuple consisting of a timestamp $t_a$ (based on the local clock
		of node $\mathcal{C}$), the public key 
\end{description}

\section*{Preliminary Assumptions}
\begin{enumerate}
	\item We use the terms ``node'' and ``actor'' interchangeably. We mean for this
		interchangeability to reflect our assumption that each communicating
		node in the network corresponds to exactly one person, and that the
		computational power is roughly equal across all nodes.
	\item All communications between nodes are direct (i.e. without any intermediary
		router, peer, etc.) through a local wireless protocol like WiFi 2.4/5 or
		Bluetooth.
	\item All nodes are subject to the same maximum number of neighbors with whom
		connections can be maintained for an extended period of time. These
		connections, of course, are only possible iff two nodes are within
		sufficient geographic proximity for a nontrival number of seconds.
	\item We assume that any connection between a pair of nodes $A,B$ is transient;
		that is, we assume that any connection between $A$ and $B$ will eventually
		be broken. It is therefore plausible---even probable, and certainly our
		intention---that this result in discrepancies between local blockchains
		of participating nodes, such that the global blockchain effectively
		``forks.''
\end{enumerate}

\section*{Protocol Outline}

\end{document}
