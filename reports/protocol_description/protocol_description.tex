\documentclass{article}
\usepackage[margin=1in]{geometry}

\usepackage[utf8]{inputenc}
\usepackage[T1]{fontenc}

\usepackage{amsmath}
\usepackage{amssymb}
\usepackage{amsthm}

\usepackage[moderate]{savetrees}

\title{Frequent-Collision Blockchains for Local Geographic Authentication}
\author{Ryan Robinett and Tiago Royer}
\date{9 Nov 2019}
\begin{document}

\maketitle

\section*{Preliminary Definitions}
\begin{description}
	\item[\textbf{Public Key:}] If $C$ is a node in the network with unique public key
		$\mathcal{C}$, we will refer to $C$ using both ``$C$'' and ``$\mathcal{C}$''
		interchangeably.
	\item[\textbf{Problem Proposal:}] A problem proposal $\mathcal{P}$ by
		node $C$ is a tuple of the form
		$$\mathcal{P}=(t_{\mathcal{P},C},\mathcal{C};\mathcal{P}^\prime_h,\mathcal{P}^\prime_t).$$
		Here:
		\begin{itemize}
			\item $t_{\mathcal{P},C}$ is the timestamp of $C$'s local clock upon
			creation of problem $\mathcal{P}$;
			\item $\mathcal{C}$ is the public key of node $C$;
			\item $\mathcal{P}^\prime_h$ is the header of block instance
			$\mathcal{P}^\prime$ (see below); and
			\item $\mathcal{P}^\prime_t$ is the tail of block instance
			$\mathcal{P}^\prime$. We refer to $\mathcal{P}^\prime$ as
			the \textbf{parent} of $\mathcal{P}$.
		\end{itemize}
		The tuple
		$\mathcal{P}=(t_{\mathcal{P},C},\mathcal{C};\mathcal{P}^\prime_h,\mathcal{P}^\prime_t)$
		naturally decomposes into the tuples
		$\mathcal{P}_h=(t_{\mathcal{P},C},\mathcal{C})$ and
		$\mathcal{P}_r=(\mathcal{P}^\prime_h,\mathcal{P}^\prime_t)$, which we
		refer to as the \textbf{header} and \textbf{reference tag} of $\mathcal{P}$,
		respectively.
	\item[\textbf{Block:}] A block $\mathcal{P}^*$ is a problem proposal instance
		$\mathcal{P}=(t_{\mathcal{P},C},\mathcal{C};\mathcal{P}^\prime_h,\mathcal{P}^\prime_t)$
		together with the tuple $\mathcal{P}_t=(t_{\mathcal{P},S},\mathcal{S},n_{\mathcal{P},S})$.
		We refer to $\mathcal{P}^*_t$ as the \textbf{footer} of $\mathcal{P}^*$. Here:
		\begin{itemize}
			\item $n_{\mathcal{P},S}$ is the nonce proposed by node $S$ which
				satisfies the hash of $\mathcal{P}_h$ (we call this ``the''
				nonce of $\mathcal{P}^*$, even though in theory many nonces
				would satisfy the inequality determine by $\mathcal{P}$'s
				hash);
			\item $\mathcal{S}$ is the public key of $S$, $S$ being the node
				which presented the solution/nonce $n_{\mathcal{P},S}$ of
				problem proposal $\mathcal{P}$; and
			\item $t_{\mathcal{P},S}$ is the timestamp of $S$'s local clock
				upon discovering nonce $n_{\mathcal{P},S}$.
		\end{itemize}
		By this construction, $\mathcal{P}^*$ can be understood as the tuple
		$(\mathcal{P}_h,\mathcal{P})r,\mathcal{P}^*_f)$, or as the tuple
		$(\mathcal{P},\mathcal{P}^*_f)$. On short time scales, two different
		nodes $S,S^\prime$ might concurrently propagate distinct nonces
		for $\mathcal{P}$ through the network, causing a global fork. It is
		necessary, in this context, to distinguish between the resulting blocks
		$\mathcal{P}^*,\mathcal{P}^{\prime *}$ when referring to the network
		beyond the scope of a single local chain (see below). However, when we
		are talking about the scope of a single node's local chain, or we are
		talking about scenarios wherein two peers agree on the validity of a
		block $\mathcal{P}^*$, we may refer to $\mathcal{P}^*$ and $\mathcal{P}$
		interchangeably, when doing so should not cause confusion.
\end{description}

\section*{Preliminary Assumptions}
\begin{enumerate}
	\item We use the terms ``node'' and ``actor'' interchangeably. We mean for this
		interchangeability to reflect our assumption that each communicating
		node in the network corresponds to exactly one person, and that the
		computational power is roughly equal across all nodes.
	\item All communications between nodes are direct (i.e. without any intermediary
		router, peer, etc.) through a local wireless protocol like WiFi 2.4/5 or
		Bluetooth.
	\item All nodes are subject to the same maximum number of neighbors with whom
		connections can be maintained for an extended period of time. These
		connections, of course, are only possible iff two nodes are within
		sufficient geographic proximity for a nontrival number of seconds.
	\item We assume that any connection between a pair of nodes $A,B$ is transient;
		that is, we assume that any connection between $A$ and $B$ will eventually
		be broken. It is therefore plausible---even probable, and certainly our
		intention---that this result in discrepancies between local blockchains
		of participating nodes, such that the global blockchain effectively
		``forks.''
\end{enumerate}

\section*{Protocol Outline}

\end{document}
