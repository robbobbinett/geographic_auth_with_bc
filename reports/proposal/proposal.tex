\documentclass{article}
\usepackage[margin=1in]{geometry}

\usepackage[utf8]{inputenc}
\usepackage[T1]{fontenc}

\usepackage{amsmath}
\usepackage{amssymb}
\usepackage{amsthm}

\usepackage[moderate]{savetrees}

\usepackage[backend=bibtex,style=ieee,natbib=true]{biblatex}
\addbibresource{blockchain.bib}
\usepackage[autostyle=true]{csquotes} % Required to generate language-dependent quotes in the bibliography

\title{Frequent-Collision Blockchains for Local Geographic Authentication}
\author{Ryan Robinett and Tiago Royer}
\date{17 Oct 2019}
\begin{document}

\maketitle

\section*{Problem Motivation}

The problem of authenticating users based on geographical location
has been considered for over 20 years~\cite{denning_1996}.
The authors of \cite{denning_1996} consider the problem of finding
the location of an intruder;
their approach easily generalizes to such applications as
geographically restricted broadcasts~\cite{gdpr}.

Beyond the scope of simple geographical authentication, however,
are applications in which both geographical data and \textit{post facto}
temporal data are useful, such as
proving someone was or was not present in a certain vecinity within a
certain timeframe.
This is most useful in cases of conflicting information or in cases sensitive to
the falsification of geo-temporal data.

Were geo-temporal data for individual persons to be continually produced and stored,
a centralized solution for managing these data would be far from ideal.
In addition to privacy concerns,
centralized solutions create a single point of failure.
Further,
a centralized solution is susceptible to the falsification of data
by the central node.
In the first example below,
the government could try to coerce the centralized node
to censor or misrepresent data about the protester.
In order to preclude these weaknesses of endemnic to centralized solutions,
a decentralized solution is necessary.

\section*{Use Cases}

\begin{itemize}
	\item A protester is found dead in a location distant from where the protest happened.
		Police argue the person was not involved in the protest
		and died from unrelated causes.
		Other protesters can use the network to contest police claims
		and show that this specific protester was indeed present in the protest.

	\item A person from a racial minority is being accused of committing a crime.
		This person can use the network to show that the accusation
		is incompatible with the locations that the person has been
		during the time the crime happened.

	\item Some expensive lab equipment disappeared overnight.
		A student not involved with this situation
		can prove her innocence by showing her location during that night
		is far away from the lab.

	\item A professor forgot to take attendance for a class.
		Each student can prove if they were in class
		even after the class has ended.

	\item BBC wants to provide exclusive content for people located in London.
		People can prove that they indeed are that geographical location
		in order to have access to this content.
\end{itemize}


\section*{Approach to the Problem}
Blockchain networks behave as probabilistic state machines, as no nodes in the
network (i.e. no node within the graph corresponding to an individual state) has
general knowledge of the rest of the network/state and must act probabilistically
based on certain assumptions about the network\cite{saito_2016}. Despite this
state machine construction, blockchain networks strive to mitigate the
possibility of a network fork as much as possible. In the case of Bitcoin, this
avoidance has been successful since the blockchain's inception\cite{decker_2013}.
Bitcoin avoids forks and maintains a single long-term history of transactions
(i.e. the longest path from the root node to the leaf on the longest branch), as
well as controls the rate of growth of the blockchain, by carefully moderating
the rate of block creation.

While the Bitcoin network moderates block creation so as to create a single,
decentralized ledger for all Bitcoin transactions for all of time, we seek to use
the blockchain dynamics described in \cite{decker_2013} to create a
blockchain-like decentralized ledger which records close-proximity peer-to-peer
handshakes as transactions and which creates blocks of these transactions at such
a rate as to force the network to fork into local networks. The intent is to
create a decentralized ledger of peer-to-peer interactions that
\begin{enumerate}
	\item creates blocks and transactions at such an exacerbatory rate that
		it is unfeasible for any node to cache a chain of blocks for
		longer than a period of one or two days; and
	\item creates blocks and transactions at such an exacerbatory rate that
		users are incentivized to only cache blocks necessary to verify
		handshakes they have made over the cache-life of a block.
\end{enumerate}
We believe that this will allow people to create a robust ledger, whose
credibility is granted by mass consensus and mutual distrust, of their temporo-geographic
location over short periods of time, in such a way that makes long-term storage of these
geographic data untenable by third parties while making sure short-term
information about location is agreed upon by several nodes in the network.

\subsection*{Potential tradeoffs}

Forcing the protocol to produce a large quantity of blocks
makes it harder for a third party to store blocks for multiple users in a long-term fashion,
but it also makes it harder to preserve data about geographical location
that would be necessary, say, in the use case of providing an alibi.
This is a tension between protocol resiliance and data preservation.

\section*{Milestones and Dates}
\subsection*{Project Deliverables:}
\begin{itemize}
	\item Write a Python class that will allow us to simulate
		blockchains with arbitrary settings, subprotocols, etc.
		Settings include:
	\begin{itemize}
		\item Protocol for initiating a handshake
		\item Protocol for propagating handshake through network
		\item Rate of block creation
	\end{itemize}
	\item Write a Python class that will allow us, given a simulated
		blockchain, to implement it (probabilistically) with
		various ``populations'' of geometrically dispersed users
	\begin{itemize}
		\item i.e. a person is a unique ID
			together with a location in $\mathbb{R}^2$.
		\item interactions between persons A and B are
			a function of their distance (and possibly
			history of misbehavior)
		\item will also write in nodes that behave adversarially
			and nodes that behave negligently
	\end{itemize}
	\item Create strong mathematical statements about
		the behaviors of frequent-collision blockchains,
		with applications to local geographic identification
		in mind.
	\item Create strong mathematical statements about our chosen
		protocol.
\end{itemize}

\subsection*{Project Deadlines:}
\begin{description}
	\item[Oct 31st:] Have Python framework written for simulating different
		blockchain protocols.
	\item[Nov 19th:] Turn in first report; have protocol fully defined and
		simulated.
	\item[Dec 1st:] Prove robustness of protocol to first-order adversarial
		attacks; prove that blockchain holds desired standing properties
		consistently.
	\item[Dec 10th:] Final report and analysis.
\end{description}

\section*{Requested Resources}
At worst, we might need some extra computing power to run some large stochastic
simulations. As long as we have access to this, we should be fine.

\printbibliography[title={Bibliography}]

\end{document}
