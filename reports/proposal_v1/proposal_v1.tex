\documentclass{article}
\usepackage[utf8]{inputenc}
\usepackage{amsmath}
\usepackage{amssymb}
\usepackage{amsthm}

\title{Frequent-Collision Blockchains for Local Geographic Authentication}
\author{Ryan Robinett and Tiago Royer}
\date{10 Oct 2019}
\begin{document}

\maketitle

\section*{Problem}
WE NEED TO DO THIS.

\section*{Motivation}
WE NEED TO DO THIS.

\section*{Proposed Solution}
WE NEED TO DO THIS.

\section*{Milestones and Dates}
\subsection*{Project Outline:}
\begin{itemize}
	\item Minimally necessary:
	\begin{itemize}
		\item Write a Python class that will allow us to simulate
			blockchains with arbitrary settings. Settings include:
		\begin{itemize}
			\item Protocol for initiating a handshake
			\item Protocol for propogating handshake through network
			\item Rate of block creation
		\end{itemize}
		\item Write a Python class that will allow us, given a simulated
			blockchain, to implement it (probabilistically) with
			various ``populations'' of geometrically dispersed users
		\begin{itemize}
			\item i.e. a person is a unique ID together,
				together with a location in $\mathbb{R}^2$.
			\item interactions between persons A and B are
				a function of their distance (and possibly
				history of misbehavior...?)
			\item Should we write in adversarial and negligent nodes?
		\end{itemize}
		\item Create strong mathematical statements about
			the behaviors of frequent-collision block chains,
			with applications to local geographic identification
			in mind. I imagine some of this will be led by our
			manipulation of variables on paper,
			and that the rest will be led by our empirical
			observations from the simulations.
	\end{itemize}
\end{itemize}

\subsection*{Project Deadlines:}
\begin{description}
	\item[Oct 31st:] Have Python framework written for simulating different
		blockchain protocols.
	\item[Nov 19th:] Turn in first report; have protocol fully defined and
		simulated.
	\item[Dec 1st:] Prove robustness of protocol to first-order adversarial
		attacks; prove that blockchain holds desired standing properties
		consistently.
	\item[Dec 10th:] Final report and analysis.
\end{description}

\section*{Requested Resources}
At worst, we might need some extra computing power to run some large stochastic
simulations. As long as we have access to this, we should be fine.

\end{document}

