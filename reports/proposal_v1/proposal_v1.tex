\documentclass{article}
\usepackage[utf8]{inputenc}
\usepackage{amsmath}
\usepackage{amssymb}
\usepackage{amsthm}
\usepackage[moderate]{savetrees}
\usepackage[backend=bibtex,style=ieee,natbib=true]{biblatex} % Use the bibtex backend with the authoryear citation style (which resembles APA)
\addbibresource{blockchain.bib} % The filename of the bibliography
\usepackage[autostyle=true]{csquotes} % Required to generate language-dependent quotes in the bibliography

\title{Frequent-Collision Blockchains for Local Geographic Authentication}
\author{Ryan Robinett and Tiago Royer}
\date{10 Oct 2019}
\begin{document}

\maketitle

\section*{Problem}

The goal is to have a descentralized network
that can provide historical geographical authentication.
Consider the following use cases.

\begin{itemize}
	\item A protester is found dead in a location distant from where the protest happened.
		Police argues the person was not even involved in the protest
		and died for unrelated causes;
		other protesters can use the network to contest police claims
		and show that this specific protester was indeed involved in the protest.

	\item A person from a racial minority is being accused of committing a crime.
		This person can use the network to show that the accusation
		is incompatible with the locations that the person has been
		during the time the crime happened.

	\item Some expensive lab equipment disappeared overnight.
		A student not involved with this situation
		can prove this by showing their location during that night
		is far away from the lab.

	\item A professor wants an attendance sheet for a class.
		The students can prove they were in class
		even after the class has ended.

	\item BBC wants to provide exclusive content for people located in London.
		People can prove that they indeed are that geographical location
		in order to have access to this content.
\end{itemize}

\section*{Motivation}
This is a citation\cite{decker_2013}.

Besides simple geographical authentication,
there are situations in which historical data is useful,
to prove someone was or was not present in a certain situation.

Centralized solutions create a single point of failure.
Furthermore,
the reliability of the data can be compromised;
in the first example,
the government could try to coerce the centralized node
to censor or misrepresent data about the protester.
Having a decentralized solution eliminates this possibility.

\section*{Approach to the problem}

The idea is to adapt the concept blockchain to allow for multiple concurrent chains,
each of them representing a geographical location.

\section*{Milestones and Dates}
\subsection*{Project Deliverables:}
\begin{itemize}
	\item Write a Python class that will allow us to simulate
		blockchains with arbitrary settings, subprotocols, etc.
		Settings include:
	\begin{itemize}
		\item Protocol for initiating a handshake
		\item Protocol for propogating handshake through network
		\item Rate of block creation
	\end{itemize}
	\item Write a Python class that will allow us, given a simulated
		blockchain, to implement it (probabilistically) with
		various ``populations'' of geometrically dispersed users
	\begin{itemize}
		\item i.e. a person is a unique ID together,
			together with a location in $\mathbb{R}^2$.
		\item interactions between persons A and B are
			a function of their distance (and possibly
			history of misbehavior...?)
		\item will also write in nodes that behave adversarially
			and nodes that behave negligently
	\end{itemize}
	\item Create strong mathematical statements about
		the behaviors of frequent-collision block chains,
		with applications to local geographic identification
		in mind.
	\item Create strong mathematical statements about our chosen
		protocol.
\end{itemize}

\subsection*{Project Deadlines:}
\begin{description}
	\item[Oct 31st:] Have Python framework written for simulating different
		blockchain protocols.
	\item[Nov 19th:] Turn in first report; have protocol fully defined and
		simulated.
	\item[Dec 1st:] Prove robustness of protocol to first-order adversarial
		attacks; prove that blockchain holds desired standing properties
		consistently.
	\item[Dec 10th:] Final report and analysis.
\end{description}

\section*{Requested Resources}
At worst, we might need some extra computing power to run some large stochastic
simulations. As long as we have access to this, we should be fine.

\printbibliography[title={Bibliography}] % Print the bibliography, section title in curly brackets

\end{document}

