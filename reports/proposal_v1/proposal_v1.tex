\documentclass{article}
\usepackage[utf8]{inputenc}
\usepackage{amsmath}
\usepackage{amssymb}
\usepackage{amsthm}

\title{Frequent-Collision Blockchains for Local Geographic Authentification}
\author{Robinett, Ryan and Tiago Royer}
\date{10 Oct 2019}
\begin{document}

\maketitle

\section*{Problem}
WE NEED TO DO THIS.

\section*{Motivation}
WE NEED TO DO THIS.

\section*{Proposed Solution}
WE NEED TO DO THIS.

\section*{Milestones and Dates}
We need to sit down after class on Thursday and write this out concretely. Right now, I am thinking the following tasks will be necessary:
\begin{itemize}
	\item Minimally necessary:
	\begin{itemize}
		\item Write a Python class that will allow us to simulate blockchains with arbitrary settings. Settings include:
		\begin{itemize}
			\item Protocol for initiating a handshake
			\item Protocol for propogating handshake through network
			\item Rate of block creation
		\end{itemize}
		\item Write a Python class that will allow us, given a simulated blockchain, to implement it (probabilistically) with various ``populations'' of geometrically dispersed users
		\begin{itemize}
			\item i.e. a person is a unique ID together, together with a location in $\mathbb{R}^2$.
			\item interactions between persons A and B are a function of their distance (and possibly history of misbehavior...?)
			\item Should we write in adversarial and negligent nodes?
		\end{itemize}
		\item Create strong mathematical statements about the behaviors of frequent-collision block chains, with applications to local geographic identification in mind. I imagine some of this will be led by our manipulation of variables on paper, and that the rest will be led by our empirical observations from the simulations.
	\end{itemize}
	\item If we do a real implementation (i.e. outside Python simulation):
	\begin{itemize}
		\item Implement blockchain (with fixed parameters) in something like ROS
		\item Perform experiments with volunteers
		\item Analyze and visualize data
	\end{itemize}
\end{itemize}

\section*{Requested Resources}
\begin{itemize}
	\item At worst, we might need some extra computing power to run some large stochastic simulations...?
	\item If we wanted to, we could experiment with cheap wireless hardware and try to perform small-scale experiments.
\end{itemize}

\end{document}

